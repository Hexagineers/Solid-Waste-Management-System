\documentclass{article}
\setcounter{secnumdepth}{5}
\setlength{\textwidth}{17cm}
\setlength{\textheight}{9in}
\setlength{\topmargin}{-1.5cm}
\setlength{\oddsidemargin}{0in}
\setlength{\evensidemargin}{0in}
\usepackage{textcomp}
\usepackage{booktabs}
\usepackage{fancyhdr}
\usepackage{times}
\usepackage{tikz}
\usepackage{amsmath}
\usepackage{tabulary}
\usepackage{pgfgantt}
\usepackage[utf8]{inputenc}
\usepackage{tikz}
\usetikzlibrary{shapes.geometric, arrows}
\tikzstyle{arrow} = [thick,->,>=stealth]
\usepackage{graphicx}
\graphicspath{ {Images/} }
\usepackage{verbatim}
\usetikzlibrary{arrows,shapes}
\usepackage{adjustbox}
\usepackage{forest}
\usepackage{tikz-qtree}
\usepackage{soul}
\pagestyle{fancy}
\rhead{CSE\hspace{\labelsep}\textbullet\hspace{\labelsep} Autumn 2017}
\lhead{\textbf{CS302} Software Engineering}
\cfoot{\thepage}
\usepackage{indentfirst}
\usepackage{graphicx}
\graphicspath{ {Images/} }
\setcounter{tocdepth}{5}
\usepackage{hyperref}
\hypersetup{
colorlinks=true,
linkcolor=blue,
filecolor=blue,
urlcolor=blue,
}
\title{\textbf{CS302}\\\HUGE Software Engineering\\
\LARGE CSE\hspace{\labelsep}\textbullet\hspace{\labelsep} Autumn 2017
}
\author{\textbf{CS 03} Hexagineers}
\begin{document}
\maketitle
\line(1,0){450}
\begin{center}
\Huge\textbf{Solid Waste Management System}\\
\Large \textbf{Requirement Analysis Report}
\end{center}
\newpage
\tableofcontents
\newpage
\section{Preface}
\par The purpose of this document is to understand the key requirements of various stakeholders involved in this project work. To begin with, we rolled survey forms, conducted interviews and met the stake holders in order to understand the current scenario and their requirements.
\par This document was prepared and reviewed by all the group members.
\newpage
\section{Introduction}
\par Solid Waste Management System is a project idea where we are trying to manage dry waste by collecting it from people and giving recycled, finished products back to the people. This is a concept we came up with where we believe that the goods could be efficiently recycled to give back to people various finished products available from the industries.
\par Apart from this, we also plan on providing know-how videos on how to reuse certain waste items available at home to make something productive.Thus, apart from benefitting the end users, the project also lays a step towards sorting out the big environment issue prevailing in India, i.e. lack of Effective Waste Management techniques.
\section{Market Research}
\par Solid Waste Management System is a pretty innovative concept. Although there are some companies which work in this field of dry waste management with similiar objective as ours. However, the domain of these companies are limited to just a specific type of dry waste.
We, on the other hand, are working on recycling all kinds of dry waste that has high convertability.
\subsection{The Stakeholders}
\par The stakeholders involved are:
\begin{itemize}
\item We, the project managers;
\item People;
\item Kabadiwalas; and
\item Recycling Industries.
\end{itemize}
\par To know the expectations from the stakeholders, we decided to conduct surveys for the direct stakeholders, i.e, the people, the kabadiwalas and the recycling industries.
\subsubsection{{People}}
\par We have surveyed about the solid waste management System in  Gandhinager. The response was pretty good. We received 60 responses of individual families in Gandhinagar. Rolling out form this way helped us to achieve the motto that our survey should comprise of sample that is evenly distributed. The survey details are as follows:
\begin{itemize}

\item What is your age group?\\
\includegraphics[scale=0.25]{Agegroup.png}
\newpage
\item How many people currently live in your house?\\
\includegraphics[scale=0.35]{People.png}
\item What is your major occupation?\\
\includegraphics[scale=0.35]{Occupation.png}
\item What type of solid waste do you mostly generate in your home or office/workspace? (Check one or more options)\\
\includegraphics[scale=0.35]{wastegenerated.png}

\item Do you sell waste products to scrap dealer (Kabadiwaala)?\\
\includegraphics[scale=0.35]{Kabadiwala.png}
\newpage
\item Approximately, how many kilograms of solid waste do you sell to the Scrap Dealer in a month?\\
\includegraphics[scale=0.35]{KG.png}
\item Do you buy products from an e-commerce website like Flipkart  ?\\
\includegraphics[scale=0.35]{Ecommerce.png}
\item Would you be interesting in getting Cash+Credits for selling to Kabdiwaala?\\
\includegraphics[scale=0.35]{Buying.png}
\item Would you be interested in buying recycled products from our website using your credits?\\
\includegraphics[scale=0.35]{Points.png}
\end{itemize}

\newpage
\par The survey results indicate that majority of the people are interested in this concept.Hence, it seems a good sign for our project. People are interested to exchange their solid waste in exchange for Cash+Credits, which they can redeem on our online e-commerce store for now. Redemption of Credits on other e-commerce websites will be introduced later on subject to feasibility.
\subsubsection{Kabadiwalas}
\par Following is the first-hand summary of the interview after the team members talked to the respective stakeholder.
\par \textit{ In our visit to the Scrap dealers place at Pethapur, Gandhinagar, we gathered information regarding the various types of scrap that is collected by them and what they do to the collected scrap later on.}
\par \textit{Our first stop was at a Scrap dealer dealing with plastic items only. He would get the scrap from several small scrap collectors that collect scrap from homes. The scrap would then be segregated there. The recycling industries from Ahmedabad would collect the scrap from time to time via trucks.However, he did not have any major problem with the system he was dealing with and even hesitated to provide us any extra information regarding his revenues and turnover.}
\par \textit{The next scrap dealer we met dealt with the rubber scraps. He would mainly collect it from the Vehicle repair shops nearby. The scrap is sent to Ahmedabad based industries where the rubber is recycled to extract the oil from it. The average transaction is around 200-300kg/month. }
\par \textit{Our next visit was to a scrap collector who himself collects from homes and sells the scrap to those in need. He doesn't give the collection to a bigger Scrap Dealer. During our conversation with the scrap dealers he told us that the problem that he faces is that he has no idea in advance as to where he would get the scrap from ? There are times when he finds difficult to collect waste items. So, he seemed very interested in our app idea and felt it would be actually easier to collect scrap via customer requests on mobile.This would prevent him from doing extra labour work and ease his work in scrap collection. It would also save him time when he knows in advance the destination where he needs to go to collect scrap.}
\par \textit{Our final stop was a scrap dealer dealing with the glass products. He generally collects the scrap from local shops and vendors for around Rs.1.5 - Rs.2 per bottle. He sells the scrap to both industries and retailers. However, sometimes due to less amount of scrap, the industries do not buy scrap from them. This leads to a situation where scrap lies around for a long time. He too felt good about the app idea and wished us luck in achieving our goals. His problems were similar to that addressed by the scrap dealer above.}
\subsubsection{Recycling Industries}
While interviewing the key personnel in the recycling industries, we covered questions that answered the feasibility of the idea, the convertibility rate of the ideas and the implmentability and scalability of the idea. Though they were bound by instructions of not giving out too many details, we got the answer to some of our key concerns. We talked to three companies based in Ahmedabad.

\begin{itemize}
    \item In our visit to Ahmedabad we have talked with one of the managers of \textbf{Let's Recycle} company. This particular Industry has there own outlets to segregate the dry waste and solid waste separately and also has recycling plants of their own. While interacting with them, we came to know they have workers who collects the waste from every possible place and also try to get in waste from kabadiwalas, who are then paid accordingly with subject to the kilos of waste delivered. The company then refines the waste in the outlets and sends them to their recycling plants once a sufficient amount of waste is collected. On an average, 20 tons of waste is sent to plants at once. In the recycling plants, recycled products like paper and bags are made form the waste like papers, leather, plastic waste etc. Solid waste which can be reused after maintenance is reused, while the other is recycled. 
Regarding our software solution, they appreciated our concept because even though there have IT solutions available for their own company, it don't include the part of kabadiwalas. With our software they can get even more waste to recycle. 

\item We also talked with \textbf{V.I.P Plastic} recycling plant over the phone. This company make machines and sell them to recycling industries. Through these machines paper and plastic bags can be made from wastes like plastics, leaves, old papers and books. Even this person liked our concept and showed interest to link us with the respective recycling plants. The industry personals have asked us to get in touch with them for actual implementation of the project.

\item We have also talked with \textbf{Perfect Plastic Plant}. They produce truck tarpaulin. Wastes like plastic scrap, old tarpaulin materials etc., are used to make tarpaulins. They liked our platform and very much interested in working with us once
implemented.

\end{itemize}

\section{Conclusion}
\par Considering the survey results and our interaction with kabadiwalas and the industries, we realized that the general public is in favor of the idea where they could buy back recycled goods in exchange for points that they would earn by giving away their solid waste. Even the kabadiwalas were interested, if they are provided with technology that will make their lives easier. Above all the project would be a big boon to the environment. So, on the basis of our requirements study we were required to implement a system where the customers directly contact the scrap dealers to sell there wastes. Also, we are to make an online E-Commerce Web store to sell the recycled goods.
\end{document}